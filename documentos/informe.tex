\documentclass[a4paper,10pt]{article}
% Paquete para la inclusion de graficos.
\usepackage{graphicx}
\usepackage[ansinew]{inputenc}
\usepackage[spanish]{babel}
% Paqiete para la inclusion de PDFs.
\usepackage{pdfpages}

\title{
	\textbf{Trabajo Pr�ctico N� 1: Conjunto de instrucciones MIPS}
}

\author{
	Sebastian Ripari, \textit{Padr�n Nro. 96.453}\\
	\texttt{sebastiandanielripari@hotmail.com}\\[2.5ex]
	Cesar Emanuel Lencina, \textit{Padr�n Nro. 96.078}\\
	\texttt{cesar\char`_1990@live.com}\\[2.5ex]
	Pablo Sivori, \textit{Padr�n Nro. 84.026}\\
	\texttt{sivori.daniel@gmail.com}\\[2.5ex]
	\normalsize{2do. Cuatrimestre de 2017}\\
	\normalsize{66.20 Organizaci�n de Computadoras  $-$ Pr�ctica Jueves}\\
	\normalsize{Facultad de Ingenier�a, Universidad de Buenos Aires}\\
}

\begin{document}

\maketitle
\thispagestyle{empty}   % quita el n�mero en la primer p�gina


\begin{abstract}
Se implemento un programa que realiza el calculo del maximo comun divisor y del minimo comun multiplo,
mediante el uso del Algoritmo de Euclides. Para la implementacion del algortimo, se utilizo el lenguaje C, y con la particularidad de las funciones matematicas fueron llevadas a cabo usando Assembler de MIPS. Por ende la compilacion del programa, comprende el linkeo de estas funciones en Assembler.
\end{abstract}

\section{Introducci�n}
El \textit{algoritmo de Euclides} es un m�todo antiguo y eficaz para calcular el m�ximo com�n divisor (MCD). Fue originalmente descrito por Euclides en su obra Elementos. El \textit{algoritmo de Euclides extendido} es una ligera modificacion que permite ademas expresar al maximo comun divisor como una combinacion lineal.

\section{Este es el T�tulo de una Secci�n}

Texto de la secci�n...

\subsection{Este es el T�tulo de una Subsecci�n}

Texto de la subsecci�n...


\section{Este es el T�tulo de Otra Secci�n}

Texto de la otra secci�n. En la figura~\ref{fig001} se muestra un ejemplo de c�mo presentar las ilustraciones del informe.

\begin{figure}[!htp]
\begin{center}
%\includegraphics[width=0.5\textwidth]{fig001.eps}
\caption{Facultad de Ingenier�a $-$ Universidad de Buenos Aires.} \label{fig001}
\end{center}
\end{figure}


\subsection{Este es el T�tulo de Otra Subsecci�n}

Texto de la otra subsecci�n...


\section{Conclusiones}

Se present� un modelo para que los alumnos puedan tomar como referencia en la redacci�n de sus informes de trabajos pr�cticos.


\begin{thebibliography}{99}

\bibitem{INT06} Algoritmo de Euclides, https://es.wikipedia.org/wiki/Algoritmo\verb|_de_|Euclides.

\end{thebibliography} 

\end{document}
